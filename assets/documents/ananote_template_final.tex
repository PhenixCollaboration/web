\documentclass[12pt]{article}
\usepackage{graphicx}
\topmargin = -15mm
\textheight = 230mm
\textwidth = 155mm
\evensidemargin = 0mm
\oddsidemargin = 0mm
\setcounter{topnumber}{300}
\setcounter{bottomnumber}{300}
\setcounter{totalnumber}{300}

\begin{document}

\begin{titlepage}
\begin{center}
 {\LARGE Template for PHENIX Analysis Notes
  } \\
\vspace{15mm}
Gabor David$^{\mbox{a}}$, Yasuyuki Akiba$^{\mbox{b}}$

\vspace{1cm}

$^{\mbox{a}}$ SBU / BNL \\
$^{\mbox{b}}$ RIKEN  \\
\end{center}

\vspace{2cm}
\date{today}



\vspace{2cm}

%\begin{document}

%\maketitle

\begin{abstract}

This write-up gives some guidelines/suggestions how to
structure analysis notes, what information to include and to what
depth.  It is not meant to be a fill-out form or checklist, the more
so, since no two analyses are similar.  Instead, it's trying to convey
the spirit in which notes should be written.  Namely, always ask
yourself: ``Had I not been doing this analysis myself, would I be able
to reproduce it and come to the exact same results based solely upon 
what's written in the analysis note?''  If the answer is negative,
please keep working on it.  A positive answer means at the very
least three things: 1/ all inputs (data, calibrations, recalibrators,
deadmaps, ect.) are clearly defined (timestamp, version) or stored 
2/ all codes, macros, scripts etc. used in the analysis
are preserved safely, long-term (not in a volatile private directory)
and 3/ the flow of the analysis is clearly described along with all
auxiliary information needed to repeat it.

\end{abstract}

\end{titlepage}


\tableofcontents

\newpage

\vspace{1.0in}
{\large {\bf General guidelines}}
\vspace{0.5in}

The overwhelming majority of PHENIX Analysis Notes are accompanying
new physics results extracted from data.  Before making them public,
as preliminary or final, the submission of a comprehensive analysis
note is a formal requirement, too.  In reality, the amount and depth
of information in the actual notes is quite disparate.  While there's
no way (and no point) to squeeze notes of the most diverse content
into a single ``checklist'', or ``fill-out form'', we think some
minimum should be adhered to.  This template is attempt to guide you
what this minimum should be (more information is always welcome!).
Also, while writing your note, you should adhere to the 
following principles.

\begin{itemize}
\item{{\bf Reproducibility.}
  The purpose of an analysis note is to ensure that our published
  results are reproducible by {\bf any} reasonably knowledgable PHENIX
  person. 
  Reproducibility means that at the very least you give a full
  list of runs analyzed, production tag, analysis (train) code
  location and tag, analysis macros with location, QA criteria, cuts
  applied  and procedures  described comprehensibly.  Err on the side
  of giving too much detail.  {\bf Omit} information referring to an earlier 
  note {\bf only} if that analysis did
  the {\bf exact} same thing as you {\bf and} if the description in it is
  truly comprehensive.  Even a 
  {\bf non-expert in your topic} should in principle be able to
  {\bf repeat} your analysis and come up with the {\bf same result}.}
\item{{\bf Provide your own systematic error estimates.}
  Unless you are analyzing the exact same data with the exact same
  code, cuts and procedures for the exact same physics quantity (but
  then, why would you do that???) your systematic errors are {\bf not}
  the same as your buddy's.  If you estimated them independently and
  came up with the same number - that's fine, but show (prove) it!  
  Statements like ``X got this number so this is my error, too'' are not 
  acceptable.}
\item{{\bf Explain differences from previous analyses.}
  If a similar analysis has already been done before, and now you
  come up with different results or systematic errors, include a
  section in which you analyze/explain
  why did things change (hopefully improve...)}
\item{{\bf Point out problems.}
  Be honest.  Saying that you didn't find the solution for a
  particular problem (and actually pointing out that problem instead
  of glossing over it) is both honest, ethical - and useful for the
  collaboration.  Pretending that it doesn't exist, or that you solved
  it, even when you yourself know that's not true is the exact
  opposite. - If you found a mistake in an earlier analysis note,
  point it out, politely!}
\item{{\bf Start early.}
  Ideally analysis notes are written {\bf in parallel with the
    actual analysis}, or at least when you are no more than halfway
  through.  Yes, that means sometimes re-writing text,
  correcting numbers a dozen times, but at least at the end what you
  wrote down will be correct - which is not necessarily the case when you
  just finished your analysis on time to meet a deadline but for
  approval you still have to submit an analysis note...}
\item{{\bf Make your note readable}.  
  Give meaningful section (subsection...) titles and
  make a table of contents.  Some otherwise excellent notes are
  lacking this ``feature''.  Ever got frustrated trying to find two
  relevant sentences or a crucial number in 30-60-170+ pages of
  unstructured and un-indexed text? 
  Also, since analysis notes are often long and sometimes hard to
  navigate, it is a good practice to give the exact location (pages)
  {\it  within} a reference (``as described on pp. n-m of ANxxx'').
  You actually looked it up, right?  Then this shouldn't be a burden.}
\item{{\bf Catch the spirit.}  Analyses are quite complex, and it is
  hard, if not impossible, to write a set of formal rules, a
  checklist, which make sure that your analysis will indeed become
  fully reproducible.  If in your case something more is needed than
  described here, please, please, add it!  {\bf Science is
    reproducibility}, and you are a scientist, aren't you?  So please,
  act like one.}

\end{itemize}


\vspace{1.0in}
{\bf Selected ``no-no''-s}

To avoid being beaten up or lawsuits, here's the disclaimer: none of
these examples are taken from actual notes, I just made them up.
Unfortunately similar things can sometimes be found in actual notes.

\begin{itemize}
\item{``We evaluated error X by a fast MC and it was negligible''.
  {\it (Which fast MC?  What parameters, input? How did you 
    evaluate?  How much is negligible?)}}
\item{``We discarded runs with a bad RP distribution.'' 
  {\it (Quantify what's ``bad''!)}}
\item{``We extracted the signal the same way as in Run-2''.
  {\it (At the very least give a reference where the Run-2 analysis is
  described in detail.  Second, it's hard to imagine that you did the
  exact same thing.  I hope you didn't, that you learned in the
  meantime, your procedures became more sophisticated.  But then they
  are different.)}}
\item{``In this step we used NN's code/libraries''.  {\it (Using other
  people's libraries -- unless, of course, they are recognized part of
  the official framework -- is frowned upon.  If absolutely
  unavoidable, specify the exact location and version, but a better
  solution is to copy the source and make it part of your own backup
  (HPSS or CVS).  This also protects you if later NN decides to change
  something in his/her code.)}}
\item{... etc.}

\end{itemize}


\newpage

%\setcounter{section}{-1}
\label{sec:organization}

\section{Analysis Organization}
label{sec:organization}
%This is a new section that we (PHENIX PM) want you to fill out to
This is the section where you
describe how your analysis is organized.  It serves as a concise
list, a catalogue of the steps of your analysis, the location of all necessary
input and output files as well as all necessary codes, macros to do
your analysis from scratch.
It is meant to enable a third person to look later into your
analysis and have access to everything needed to reproduce your
results, if necessary.  Basically it is about preserving the
{\bf data} and {\bf software} necessary.
Detailed description and explanation of the 
individual analysis steps -- the {\bf know-how} -- 
comes in the later part of the analysis note where
we still suggest some of the information to be repeated.

This section should include the following information.
\begin{description}
\item [A flowchart of the major analysis steps with code locations]~\\
The flowchart is a birds-eye view of your analysis flow, with
references to the basic working directories.
``Code location'' here means your working directories on RCF.
Ideally these all should branch off a single {\bf base directory} (which
also makes backing up the ``snapshot'' -- see below -- in HPSS easier).
Important: if you did some of the work locally (e.g. on your laptop),
you have to {\bf migrate} the codes and relevant files {\bf to RCF} 
when you are finished, and make sure it works there, too. 
%\item [Location of your code and working directories in the RCF]~\\

Examples of directories to be specified:
\begin{itemize}
\item Directory for Taxi code
\item Directory to run Taxi
\item Directory of the Taxi output
\item Directory for Simulation code
\item Directory for Simulation output
\item Directory for analysis code and macros that analyze the Taxi output
\item Directory for analysis code and macros that analyze the Simulation output
\item Any other directory used during the analysis
\item Directory for the final data file(s) and macros to produce physics plots.
\end{itemize}
We suggest you write these directories as they are at the time you
finished your analysis, and you reference them in the later part of 
the analysis note.  After they are backed up in HPSS, part or all of
it can be removed.  
%We understand that they can be removed later. But it is not a problem, since
%we require that you should back up all of these directories to be backed up in HPSS.
 
\item [Location in HPSS of your analysis snapshot]~\\
We request that a snapshot of your analysis, namely, all of the 
directories and the files in the directories described above, are
backed up in HPSS.  
%Ideally those directories should be {\bf clean} in the
%sense that they contain {\bf all those files} that are needed to
%repeat the analysis from scratch, but {\bf only those files} --
%however, cleaning them up is not a formal requirement.  (Obviously
%cleaning up poses the danger that you delete something that's actually
%necessary for the analysis.)

%Please back up all your working directories
%described above to the HPSS.

%We request that all of your code and intermediate data files of
%the analysis should be backed up. 
Before the backup, make a good faith effort to {\bf clean up} the
directories (such that someone later looking at it isn't unnecessarily
distracted).  Log files, for instance, can be deleted.  
However, make sure that they contain {\bf all those files} that are
needed to repeat the analysis from scratch.  
{\bf Avoid symlinks!}  If you
aggregated the taxi output (e.g. with {\it haddPhenix}), you can 
make a text file of the run numbers processed (runlist), then delete the
individual run outputs.  Same for simulations.  However, keep a small
fraction of the original files (and include them in the backup).
%However, if the total amount of the Taxi output or Simulation output
%is too large, it is sufficent to back up a fraction of them.
%For example, if there are 1,000 simulation nDSTs of 1GB each (Total 1TB),
%it would be suffice to back up 100 nDSTs
Then please document the location of your analysis snapshot
in HPSS here.  

{\bf Note:} At some point in the future the Data Preservation Task
Force is planning  to centralize the location of those analysis
snapshots into one place.  In order to make this easier, in addition
to documenting the HPSS location of your snapshot in the analysis
note, please also send an email indicating this location to Chris
Pinkenburg, Maxim Potekhin, Takahito Todoroki and Gabor David.

{\bf Dependencies.}
If you are aware that your code uses libraries other than the ones
created from your own code or are part of the core PHENIX software,
please point out this fact here.  Ideally the source code for those
other libraries should also be included in your backup.

{\bf Remember}, working directories are handy, easy to
  access, but they can easily come and go, be deleted, altered, so a
  copy of the snapshot of your analysis - files, codes, macros - at
  the time of preliminary request or the final publication
  should exist in HPSS/CVS.

 If you don't know how to back up a directory tree to HPSS, you'll find
 instructions at  the end of this section.

\item [Brief summary of your analysis]~\\
State the physics observable analyzed, the basic method and the
subdetectors involved.  Then please write a brief description
(overview) of the steps of your analysis and in 
which your working directory each step is done.
These are essentially bullets explaining the flowchart and adding some
basic information, with pointers to working directories.
Detailed description comes in later sections of the analysis notes.
\item [Data Set] ~\\
\begin{itemize}
\item Run, beam species, energy, trigger(s)
\item Good run list and its location.   (Don't put the good run list 
itself here. If you want to include it in the analysis note, too, put
it in an Appendix)
\item Number of events analyzed and the corresponding number of MB 
events or integrated luminosity
\end{itemize}
\item [Taxi] ~\\
\begin{itemize}
\item Taxi code and its CVS location, including version number
\item List of recalibrators used for the analysis. Are all
recalibrators public? If there is any private recalibrator used, 
please list them and their location.
\item List of files like efficieny map or data for recalibrator
\item Taxi output
\end{itemize}
\item [Simulation] ~\\
Simulation code and any related files
\begin{itemize}
\item List of simulation code used for the analysis
\item Event generator(s), including tuning/parameter files
\item PISA setup file
\item Tuning of PISA
\item Output of simulation (PISA, PYTHIA, etc)
\item List and location of code to analyze the simulation data and its output
\item Any other pertinent information
\end{itemize}
\item [List of major intermediate analysis files] ~\\
There are several steps in the analysis from Taxi run to the final
data file. You will produce intermediate analysis file in each
step. Please list major intermediate files produced by the
analysis. Remember: reproducibility means that {\bf each} step of your
analysis process can be unambiguously traced.  For example

\begin{itemize}
\item Code to produce the raw data histogram of pi0 from Taxi output
\item ROOT file of the raw data histogram of pi0 pT distribution
\item ROOT file of PISA simulation for pi0 efficiency calculation
\item Embedding code/macro
\item Code to analyze the PISA simulation file to produce pi0 efficiency
\item ROOT file of the efficiency vs pT of pi0
\item Macro to calculate the cross section from the raw histogram and efficiency
\item Any other pertinent information
\end{itemize}
\item [Final data file and plotting macro]~\\
We now request that final physics data for figures for a journal paper and preliminary plot should be
put in ROOT files, and the figures/plots are produced by plotting macros from these final data files.
We request that all of the ROOT files of plotted data and the plotting macros are in a single
directory.
Please list the final data files, plotting macros, and their location here.
Please indicate which macro will produce with plot from which data file.\\

In addition you have to provide your {\bf final data points} with
uncertainties in {\bf text files} (ASCII), and specify their
location.  These files are needed because 1/ we have to post the data
on our public website 2/ we want to upload them in HEP databases.
(While strictly speaking this is not a requirement for preliminary
data, it is good practice to provide ASCII files from those, too.)

Example:\\
The data file and plotting macros of the preliminary request is located at
\begin{verbatim}
/phenix/u/analyzer/analysis/prelim2019.10
\end{verbatim}
\verb*|DataFile.root|  The data file of all plots requested for preliminary\\
\verb*|plot_xs_pi0.C|  Macro to produce Plot1 (cross section of pi0).\\
\verb*|plot_RAA_pi0.C|  Macro to produce Plot 2 (RAA of pi0)\\

The ROOT files of the plotted data and the plotting macros should be backed up in HPSS as a part of
the snapshot of your analysis.

We request that the directory of the ROOT files of the plotted data and the plotting macros
should be under your user directory (\verb*|/phenix/u/analyzer| ) so that it is in RAID part
of RCF filesystem. We request that this final data directory should be "frozen"
at the time of preliminary request or final paper publication, except for editing of the plotting macros
for cosmetic changes of the plots.
Important: this directory should be duplicated on the rcas disk, and
be part of your analysis snapshot, too (backed up in HPSS).

\item[How to put a directory tree in HPSS?]~\\
First make a .tar file of the directory tree, 
preferably on the PHENIX scratch area (if you don't already have a
directory there, you can create one) \\
\begin{verbatim}
cd /gpfs/mnt/gpfs02/phenix/scratch/david
\end{verbatim}
then do the equivalent of this
\begin{verbatim}
tar -cvM -f plhf1_david_taxi.tar /gpfs/mnt/gpfs02/phenix/plhf/plhf1/david/taxi
\end{verbatim}
While sitting in this (scratch) directory, log in to HPSS (hsi
command, the password is the one you use for rcf3, rcas)
\begin{verbatim}
[david@rcas2064 david]$ hsi
Warning: "Network Options" section is empty in HPSS.conf file
nd_krb_preexist_auth: No credentials cache found retrieving principal
name from cred.cache
nd_kerberos_auth: No credentials cache found on krb5_mk_req call
Kerberos Principal: david
Password for david@SDCC.BNL.GOV:
Username: david  UID: 1113  Acct: 1113(1113) Copies: 1 Firewall: off 
[hsi.5.0.2.p3 Mon Jun 29 16:29:32 EDT 2015]
? pwd
pwd0: /home/david
\end{verbatim}
You end up in your HPSS directory.  Create a subdirectory 
for the new full backup (in this case it is my old taxi directory), 
then ``mput'' the tar file in HPSS (per default ``mput'' will copy files
from the directory where you have been when you logged in to HPSS with
the hsi command):
\begin{verbatim}
? mkdir taxi_20190828
mkdir: /home/david/taxi_20190828
? cd taxi_20190828
? mput *
mput 'plhf1_david_taxi.tar'? ([Y]es,[N]o,[A]ll,[Q]uit) y
Active:   1, Queued     0 [In-flight: 193.25GB Xferred: 1.10GB 0.57%] [Done: 0 F    
\end{verbatim}
and it will keep showing you the progress how the file is uploaded in HPSS.
Since the file first goes to a huge buffer-disk, the transfer is
pretty fast.  It ends with this message, when you log out:
\begin{verbatim}
mput  'plhf1_david_taxi.tar' :
'/home/david/taxi_20190828/plhf1_david_taxi.tar' 
( 193246924800 bytes, 220612.0 KBS (cos=13))
? ls
/home/david/taxi_20190828:
plhf1_david_taxi.tar
? q
[david@rcas2064 david]$
\end{verbatim}
That's all, you are back on rcas, in the scratch directory.  Be nice
and delete the .tar file right away (don't wait for the automatic
deletion). 
\end{description}

\newpage

\section{Introduction}

{\it 
  Define the purpose and scope of your analysis.  Describe (briefly!)
  the physics you are after and how/why is the signal you extract
  relevant to it.  If this is not the first time such signal is
  analyzed, refer (briefly!) to earlier PHENIX work and point out
  what's different (method? dataset? statistics? better simulations?
  new insight in some bias?  bugs found and fixed? ...)
  Describe (briefly!) the improvements you made and the major
  unresolved issues.
}

{\it In what follows, keep in mind that your analysis has to be 
  {\bf reproducible}.  If you fulfilled requirements described in
  Sec.~\ref{sec:organization} it means that the {\bf software} environment is
  preserved.  But that's obviously not sufficient. 
  Now you have to make sure that the {\bf know-how} is
  preserved too, the ``recipees'' how to do the individual steps,
  proper sequence, dependencies are spelled out.  In some cases this
  is trivial, but it can also get very tricky, interconnected,
  iterative; you have to make sure, that a third person can
  understand, what to do with the stored software.
}

\section{Data set, QA, general cuts}

{\it
  Decribe precisely your input dataset (a table of all runs analyzed
  should appear in the Appendix~\footnote{Yes, this still leaves the
  question of segments open, but that would be overkill}): 
  data taking period, master DST/pDST production tag, train number.
  Describe your run selection (QA) criteria, preferably with 
  justification.  Plot the relevant global quantities (centrality?
  reaction plane? multiplicity? ...) from the runs passing your QA.

  If you have event selection criteria, list and justify them.  Show
  the relevant global quantities before and after selection.  If you
  use triggered data, show the turn-on curve.

  Summarizing in a table what fraction of runs/events were eliminated
  by different cuts is good practice.

  Plot live/dead areas (maps) in the relevant detectors.  Indicate if
  they influence acceptance in a $p_T$- or charge-dependent way.
  Indicate if they are fixed over the dataset or run-dependent.

}

\section{Data Analysis}


%\subsection{CVS location of the codes and macros}
%\label{CVS}

%{\it
%  Give exact location(s) - not just ``it's in CVS'' - of all your codes and
%  macros used in the analysis.  Include version number and time when
%  your final scan was done (DB contents!).  It should be ready to use:
%  if there are some special requirements to run it, please add a README.
%  Don't forget simulation codes (event generator, PISA, response, 
%  reconstruction, evaluation codes, fast MC, whatever you used).
%  Include special (parameter) files used in your simulations, if any.
%  If you used embedding code, provide its location, too.
%}

\subsection{Analysis chain}

{\it
  List the steps of your analysis (typically this will be a much
  longer list than the one in Sec.~\ref{sec:organization}) and
  point to the subsections where they are described in detail.  If you
  are doing something new/unusual, point out this fact here.
}

{\it As said before, there's no way to write up a general, mechanical
  ``checklist''.  Use your judgement and always try to think along the
  lines: ``If I were the outsider, would I fully understand what has
  been done?''  If not, keep working on the analysis note.
}

\subsubsection{Analysis step 1}

{\it
  Give a meaningful title to the subsection.  Provide a reference to
  the directory where this step has been done and what macros, scripts
  were run, with what parameters.
  Define your cuts precisely, and plot its
  effects (before/after).  For log distributions consider ratios.  If
  you make a fit, always give the precise functional form, its parameters
  and plot the data/fit ratio.
}



\subsubsection{Analysis step 2, ... n}

{\it
  In general, plot intermediate results.  It is much easier to believe
  not only your ideas, but the actual work you have done if one sees
  how your raw data evolved in the individual steps into the final,
  fully corrected results.
}

\subsubsection{Simulations}

{\it
  Your simulations should also be reproducible: provide all pertinent
  information (generator, acceptance cuts, ranges, ...).  If
  simulation output has to be modified (like an additional smearing),
  explain.  Provide plots that show how well simulations describe real
  data (like simulated and real peaks, widths).  
  Provide location and tag of the embedding code (if any).
}


\subsection{Systematic errors}

{\it 
{\bf Make your own estimate} for each error related to your own analysis,
except for ``external'' quantities you take from someone else (centrality,
reference spectrum...) - for those give exact references.  If you
claim some error cancels in a composite quantity computed from your
and an ``external'' result (like a p+p denominator in a heavy ion
$R_{AA}$) make sure and prove with a reference that the relevant parts
of the analyses were done the exact same way (often they were not!).
}

\subsubsection{Error source 1}

{\it 
  Describe the error source, explain, what type it is (A,B,C,...) and
  whether it is centrality-dependent, absolute or relative.  Explain
  how you estimated its  contribution, and whether your estimate is
  full extent, $\sigma$ or something else.  Include a plot from which
  the reader can pass his/her own judgement on your estimate.  Always
  give a specific number or upper limit (not just ``it's small'').
  Your errors should be clearly defined here (if necessary, listed in
  a  table) for all data points.

  Confessing that a particular estimate is ``soft'' is not a shame
  (even if the PWG grilles you for it).

}

\subsubsection{Error source 2, ...n}

{\it
  If possible, for $p_T$-correlated errors provide an estimate how is
  it (anti)correlated; a function is even better.  Justify it.
}

\subsubsection{Summary of systematic errors}

{\it
  Provide a table that summarizes your systematic errors.  If such
  table would become too big, you may want to select only a few
  points, representative of your entire range ($p_T$, centrality,...).


}


\subsubsection{(If relevant: changes in error estimates since earlier results)}

{\it
  If this is not the first analysis of its kind (preliminary to final,
  re-analysis of a dataset, etc.) and if your systematic errors differ
  significantly from the ones found before, discuss in detail what
  changed (method? assumptions? new insight? etc.) and why do you
  think the new estimate is better than the previous one.
}

\subsubsection{(If relevant: how to improve systematic errors in the future?)}

{\it
  This could be one of the most useful parts of your note.
  During your analysis you often got frustrated, because you couldn't
  improve on something due to lack of specific data, software
  capabilities, unresolved detector problems, etc.  Discuss them,
  particularly if they turned out to be your dominant errors, and
  suggest improvements.
  Try to be specific, beyond generalities like ``we should do more 
  simulations'' or  ``we need better hadron rejection''.
}


\section{Results}

{\it
  Plot all your results with clear captions.  If there are too many
  (say, more than 10-12 pages) break them
  up into subsections for easier reading/navigation.  If you use
  ``external'' data (like reference spectra), plot them separately,
  too.  If previous results of the same quantity exist, plot old/new,
  and if there are big discrepancies, describe briefly the reason (even
  if you discussed it already in detail in the analysis section).
}

\subsection{Discussion, comparison to theory}
{\it
  If possible, relevant and available, compare your results to theories, 
  plot and discuss the results.  This is not a paper to be published, 
  so feel free to wander off in less-than-mainstream directions as well.
  They might actually give someone an idea...
}


\section{Appendix}

\subsection{Data tables}
{\it
  No results without data tables!  List errors separately
  (statistical, type A, B, C) and the total.  If you have many tables,
  breaking them up into subsections is strongly recommended.
  Any analysis note without data tables will be immediately rejected.
}



\subsection{(Runlist)}



\begin{thebibliography}{9}

\bibitem{ref1} Your reference
{\rm http://www.phenix.bnl.gov/yourreference/ref1.pdf}


\end{thebibliography}

\end{document}



